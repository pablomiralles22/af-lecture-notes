\subsection{Aplicaciones del teorema espectral.}

\subsubsection{Teoría espectral para operadores integrales.}

\begin{definition}
  Sea $T:H\to H$ operador, con $H$ Hilbert, y sean $b\in H$ y $\lambda\in
  \mathbb{K}-\{0\} $. Queremos encontrar $u\in H$ tal que $\lambda u-Tu=b$ (P).
  Llamamos problema homogéneo (P$_h$) al caso $b=0$.
\end{definition}

\begin{theorem}[Alternativa de Fredholm]
  \label{th:fredholm}
  Dado $T$ operador compacto y autoadjunto, el problema (P) tiene solución si y
  solo si  $\dotp{b}{v}=0$ $\forall v$ solución de (P$_{\text{h}}$).
\end{theorem}

\begin{proof}
  Por Hilbert-Schmidt (\ref{th:hilbert-schmidt}), existe una base
  $\{u_n\}_1^\infty$ base hilbertiana con $Tu_n=\lambda_n u_n$ tal que
  $|\lambda_1|\ge |\lambda_2|\ge \ldots \to 0$.


  Supongamos que $u$ es solución de (P), observamos que
  \[
  (\lambda-\lambda_n)\dotp{u}{u_n} =
  \dotp{\lambda u}{u_n} - \dotp{u}{\bar{\lambda_n} u_n} =
  \dotp{\lambda u}{u_n} - \dotp{u}{\lambda_n u_n} =
  \dotp{\lambda Id u}{u_n} - \dotp{u}{Tu_n} =
  \] 
  \[
  \dotp{\lambda Id u}{u_n} - \dotp{Tu}{u_n}=
  \dotp{(\lambda Id - T) u}{u_n}=
  \dotp{b}{u_n}
  .\] 

  Si $\lambda\neq \lambda_n$ $\forall n\in N$:

  \[
  u=\lambda^{-1}(b+Tu)=
  \lambda^{-1}\left(b+\sum_{n=1}^{\infty} \lambda_n\dotp{u}{u_n}u_n\right)=
  \lambda^{-1}\left(b+\sum_{n=1}^{\infty}
  \lambda_n\frac{\dotp{b}{u_n}}{\lambda-\lambda_n}u_n\right)
  .\] 

  Considero $\alpha_n=\frac{\lambda_n}{\lambda-\lambda_n}$, y estudiemos la
  serie $\sum_{n=1}^{\infty} |\alpha_n|^2|\dotp{b}{u_n}|^2$. Como los
  $\lambda_n\to 0\neq \lambda$, $|\alpha_n|<C\in \mathbb{R}$ $\forall n\in
  \mathbb{N}$, y por lo tanto la serie queda acotada por $C^2\|b\|^2$ y
  converge, y dicho $u\in H$ existe. Además:

  \[
  Tu=
  \lambda^{-1}\left( Tb+\sum_{n=1}^{\infty} \alpha_n \lambda_n \dotp{b}{u_n}u_n \right)=
  \lambda^{-1}\left( \sum_{n=1}^{\infty} \lambda_n \dotp{b}{u_n}u_n +
                     \sum_{n=1}^{\infty} \alpha_n \lambda_n \dotp{b}{u_n}u_n \right)=
  \]
  \[
  \left( \sum_{n=1}^{\infty} \lambda^{-1}\lambda_n(1+\alpha_n) \dotp{b}{u_n}u_n
  \right)
  .\] 

  Queda comprobar que se cumple la ecuación.

\end{proof}

\begin{corollary}
  Fijado $\lambda\in \mathbb{K}-\{0\}$, si (P) tiene a lo más una solución
  $\forall  b\in H$, entonces se verifica que el operador lineal y continuo
  $(\lambda Id-T)^{-1}:H\to H$ queda bien definido. Además, la ecuación (P)
  tiene como solución $u=(\lambda Id-T)^{-1} b$.
\end{corollary}
