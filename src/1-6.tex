\subsection{Dual de un espacio de Hilbert: teorema de Riesz-Fréchet.}

\begin{theorem}[F. Riesz - M. Fréchet]\label{th:riesz}
  Sea un espacio de Hilbert $H$ y una forma lineal $f:H\to \mathbb{K}$. Las
  siguientes afirmaciones son equivalentes:
  \begin{enumerate}
    \item La forma lineal $f$ es continua.
    \item Existe un único $y\in H$ tal que $f(x)=\dotp{x}{y}$ para cada $x\in
      H$, siendo además $\|f\|=\|y\|$.
  \end{enumerate}
\end{theorem}

\begin{definition}
  Sea $X$ espacio vectorial sobre $\mathbb{K}$, $B:X \times X\to \mathbb{K}$.
  \begin{enumerate}
    \item Se dice que $B$ es bilineal si fijados $x,y$ respectivamente, $B(\cdot
      ,y),B(x,\cdot )$ son lineales.
    \item Se dice que $B$ es sesquilineal si $B(\cdot ,y)$ es lineal y
      $B(x,\cdot )$ es lineal conjugada, esto es,
      $B(x,ay)=\overline{a}B(x,y)$.
    \item Se dice que $B$ es simétrica si $B(x,y)=B(y,x)$ $\forall x,y\in H$.
    \item Se dice que $B$ es positiva si $B(x,x)\ge 0$ para cada $x\in X$.
  \end{enumerate}

  Si $X$ es normado:

  \begin{enumerate}
    \item Se dice que $B$ es acotada si $\exists M>0$ tal que $|B(x,y)|\le
      M\|x\|\|y\|$ para cada $x,y\in X$.
    \item Se dice que $B$ es fuertemente positiva si $\exists c>0$ tal que
      $B(x,x)\ge c\|x\|^2$ para cada $x\in X$.
  \end{enumerate}
\end{definition}

\begin{theorem}[Lax-Milgram]\label{th:lax-milgram}
  Sea $H$ espacio de Hilbert sobre $\mathbb{K}$ y $B$ una forma sesquilineal en
  $H$ acotada y fuertemente positiva. Entonces existe un isomorfismo de espacios
  de Hilbert $T:H\to H$, unívocamente determinado, tal que
  \[
  B(x,y)=\dotp{x}{Ty}
  ,\] 
  para cada $x,y\in H$.
\end{theorem}
