\subsection{Operadores diferenciales y soluciones débiles.}

\begin{definition}
  Dado $\Omega\subset \mathbb{R}^n$ abierto, se define
  \[
  \mathcal{D}(\Omega)=\left\{g:\Omega\to \mathbb{R}:g\text{ de clase
  }C^\infty,sop(g)=\overline{\{x:g(x)\neq 0\} } \text{ compacto}\right\} 
  .\] 
\end{definition}

\begin{definition}
  Dado el operador diferencial lineal $L=\sum_{|\alpha|\le n}a_\alpha\left(
  \frac{\partial}{\partial x} \right)^\alpha$, con los $a_\alpha\in \mathbb{K}$,
  se define el operador adjunto como $L^*:=\sum_{|\alpha|\le
  n}(-1)^{|\alpha|}\overline{a_\alpha}\left( \frac{\partial}{\partial x}
\right)^\alpha$.
\end{definition}

\begin{lemma}[Regularación]
  $\mathcal{D}(\Omega)$ es denso en $L^2(\Omega)$ para el producto escalar
  estándar.
\end{lemma}

\begin{theorem}[Gauss]\label{th:gauss}
  Dado $h\in C^(\overline{\Omega})$, si $\partial \Omega$ es suficientemente
  buena entonces
  \[
    \int_\Omega \frac{\partial h}{\partial x_j} (x)dx=
    \int_{\partial \Omega} h\cdot n_j d\theta
  .\] 
\end{theorem}

\begin{proposition}
  $\dotp{L\phi}{\psi}=\dotp{\phi}{L^*\psi}$ $\forall \phi,\psi \in
  \mathcal{D}(\Omega)$.
\end{proposition}

\begin{proof}
  Lo vemos para $L=\frac{\partial }{\partial x_j} $, bastando aplicar
  (\ref{th:gauss}) a $h=\phi\cdot \psi$, observando que por tener estas soporte
  compacto valen $0$ en la frontera, y que aunque la frontera de $\Omega$ no sea
  suficientemente buena, como el soporte es compacto siempre podemos tomar un
  abierto entre el soporte y $\Omega$ con frontera suficientemente buena.

  Por inducción se generaliza a derivadas parciales de cualquier orden, y por
las propiedades del producto escalar al caso general.
\end{proof}

\begin{corollary}
  $\forall f\in L^2(\Omega)$, si $u$ es suficientemente regular y verifica que
  $Lu=f$, entonces $\dotp{f}{\phi}=\dotp{u}{L^* \phi}$ $\forall \phi \in
  \mathcal{D}(\Omega)$.
\end{corollary}

\begin{definition}
  Si $u\in L^2(\Omega)$ verifica $\dotp{f}{\phi}=\dotp{u}{L^* \phi}$ para cada
  $\phi \in \mathcal{D}(\Omega)$, entonces decimos que es solución débil de la
  ecuación $Lv=f$.
\end{definition}
