\subsection{Teorema de la acotación uniforme o de Banach-Steinhaus.}

\begin{theorem}[Banach-Steinhaus, de la acotación uniforme]
  Sea $\{A_i: i \in I\} $ una familia de aplicaciones lineales continuas del
  espacio normado $X$ en el espacio normado $Y$ y sea
  \[
  D=\{x\in X: \sup_{i \in I} \|A_i(x)\|=\infty\} 
  .\] 
  Entonces:

  \begin{enumerate}
    \item Si $D^c$ es de segunda categoría, entonces $\sup_{i \in I}\|A_i\| <
      \infty$ y $D$ es vacío.
    \item Si $X$ es de Banach, entonces, o bien $\sup_{i \in I}\|A_i\|<\infty$,
      o bien $D$ es un $G_\delta$ denso en $X$.
  \end{enumerate}
\end{theorem}

Otra versión más sencilla, que es la que puede preguntar en el examen de teoría:

\begin{theorem}[Banach-Steinhaus, de la acotación uniforme]
  % Dados dos espacios de Banach, y una familia de aplicaciones lineales y
  % continuas entre ellos, si $\forall x\in X$ $\sup_{i \in
  % I}\|A_i(x)\|<\infty$, entonces $\sup_{i \in I}\|A_i\|<\infty$.
  Dada una familia de aplicaciones lineales y continuas $\{A_i\}_{i \in I}$
  entre un espacio de Banach $X$ y un espacio normado $Y$, si $\forall x\in X$
  $\sup_{i \in I}\|A_i(x)\|<\infty$, entonces $\sup_{i \in I}\|A_i\|<\infty$.
\end{theorem}

\begin{proof}(p.265-266, \cite{cascales2012})

  Considero los conjuntos
  \[
  D=\{x\in X: \sup_{i \in I} \|A_i(x)\|=\infty\} 
  ,\]
  \[
  D_n=\bigcup_{i \in I}\{x\in X: \|A_i(x)\|>n\} 
  .\]

  Los $D_n$ son abiertos porque cada $A_i$ y la norma son continuas, y
  $D=\bigcap_{n\in \mathbb{N}}D_n$. Por hipótesis $D=\emptyset$, y si cada
  $D_n$ fuese denso $D$ también lo sería (\autoref{th:baire}, pues $X$ es
  Banach), así que algún $D_n$ no es denso, y $\exists m\in \mathbb{N}$ con
  $Int(D_m^c)\neq \emptyset$. Considero $x\in Int(D_m^c)$, y $r>0$ con
  $B(x,r)\subset D_m^c$, de forma que $\forall y\in B(x,r)$ se tiene
  $\|A_iy\|\le m$. Considero además $C=\sup_{i \in I}\|A_ix\|<\infty$. Se puede
  comprobar entonces, para $y\in B(0,r)$:

  \[
  \|A_iy\|=\|(A_iy - A_ix) +A_i x\|\le 
  \|A_i(x-y)\|+ \|A_ix\|\le 
  m + C
  ,\] 

  y por lo tanto $\forall y\in B[0,1],i \in I$ se da $\|A_iy\|\le 2r(m+C)$, con
  lo que $\sup_{i \in I}\|A_i\|<\infty$.

\end{proof}
