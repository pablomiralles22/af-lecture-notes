\subsection{Teorema de la acotación uniforme o de Banach-Steinhaus.}

\begin{theorem}[Banach-Steinhaus, de la acotación uniforme]
  Sea $\{A_i: i \in I\} $ una familia de aplicaciones lineales continuas del
  espacio normado $X$ en el espacio normado $Y$ y sea
  \[
  D=\{x\in X: \sup_{i \in I} \|A_i(x)\|=\infty\} 
  .\] 
  Entonces:

  \begin{enumerate}
    \item Si $D^c$ es de segunda categoría, entonces $\sup_{i \in I}\|A_i\| <
      \infty$ y $D$ es vacío.
    \item Si $X$ es de Banach, entonces, o bien $\sup_{i \in I}\|A_i\|<\infty$,
      o bien $D$ es un $G_\delta$ denso en $X$.
  \end{enumerate}
\end{theorem}

Otra versión más sencilla, que es la que puede preguntar en el examen de teoría:

\begin{theorem}[Banach-Steinhaus, de la acotación uniforme]
  % Sea $(T_n)_n$ una sucesión de aplicaciones lineales continuas de un espacio de
  % Banach $X$ en un espacio normado $Y$ tal que existe $T(x)=\lim T_n(x)$ para
  % cada $x \in X$. Si $\forall x\in X$ $\sup_{i \in I}\|T_i(x)\|<\infty$,
  % entonces $\sup_{i \in I}\|A\|<\infty$.
  Dados dos espacios Banach, y una familia de aplicaciones lineales y
  continuas entre ellos, si $\forall x\in X$ $\sup_{i \in
  I}\|A_i(x)\|<\infty$, entonces $\sup_{i \in I}\|A_i\|<\infty$.
\end{theorem}

\begin{proof}
  
\end{proof}
