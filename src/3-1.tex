\subsection{Teorema de Hahn-Banach.}

\begin{definition}
  Un espacio vectorial topológico es un espacio vectorial con una topología en
  la que el producto por escalares y la suma de vectores son aplicaciones
  continuas.
\end{definition}

\begin{example}
  $\Omega \subset \mathbb{R}^n$ abierto,
  $C^\infty_0(\Omega)=\mathcal{D}(\Omega)$ son las funciones $C^\infty$ con
  soporte compacto. Fijado un compacto $K\subset \Omega$, se puede fijar la
  topología de la convergencia uniforme sobre $K$ sobre el espacio
  $\mathcal{D}_K(\Omega)\subset \mathcal{D}(\Omega)$ de las funciones con
  soporte contenido en $K$.

  Cualquier abierto de $\mathbb{R}^n$ se puede poner como unión numerable de
  compactos crecientes, de forma que $\mathcal{D}(\Omega)=\cup
  \mathcal{D}_{K_n}(\Omega)$. Cada $\mathcal{D}_{K_n}$ es un espacio de Banach
  con la topología anterior, y se puede construir en $\mathcal{D}(\Omega)$ la
  topología límite inductivo, que es la más fina tal que las inclusiones de los
  $\mathcal{D}_{K_n}(\Omega)$ son continuas. Este espacio es localmente convexo,
  y además se puede definir un dual $\mathcal{D}(\Omega)^*$.
\end{example}

\begin{theorem}[Hahn-Banach, versión geométrica]
  Dado $X$ espacio de Banach, $A \subset X$ convexo y cerrado, $x_0\not\in A$,
  entonces existe $f\in X^*$ y $H=\{x\in E:f(x)=\lambda\} $ separa $A$ de $x_0$.
\end{theorem}

\begin{corollary}
  Todo conjunto cerrado y convexo en un espacio de Banach es intersección de
  semiplanos.
\end{corollary}

\begin{theorem}[Hahn-Banach, versión analítica en un espacio de Hilbert]
  Sea $H$ espacio de Hilbert, y $F\subset H$  subespacio cerrado. Sea $f:F\to
  \mathbb{K}$ lineal y continua, entonces existe una única $\tilde f:H\to
  \mathbb{K}$ una aplicación lineal y continua extensión de $f$.
\end{theorem}

\begin{proof}
  $F$ cerrado en Hilbert implica Hilbert, por el teorema de Riesz $\exists v_f
  \in F$ tal que $f(x)=\dotp{x}{v_f}$ $\forall x\in F$, y $\|v_f\|=\|f\|$.
  Definiendo $\tilde f(x)=\dotp{x}{v_f}$ $\forall x \in H$ es fácil comprobar la
  veracidad de las afirmaciones.
\end{proof}

\begin{theorem}[Hahn-Banach, versión geométrica en un espacio de Hilbert real]
  Sea $H$ espacio de Hilbert real, y $C\subset H$ convexo cerrado y $x_0 \not\in C$.
  Entonces $\exists f:H\to \mathbb{R}$ lineal y continua tal que
  $f(x_0)>\alpha>f(c)$ $\forall c \in C$.
\end{theorem}

\begin{proof}
El problema 1.52 dice que $y\in C$ cumple que $\|x_0-y\|=dist(x_0,C)$ si y solo
si $Re\left( \dotp{x_0-y}{w-y} \right) \le 0$ para cada $w\in C$. En el caso
real no hace falta usar la parte real. Llamo $y_0$ al vector que cumple eso, y
defino $f:H\to \mathbb{R}$ dada por $\dotp{\cdot }{x_0-y_0}$, se deja como
ejercicio comprobar las afirmaciones.
\end{proof}

\begin{lemma}
  Sea $X$ espacio vectorial real y $p:X\to \mathbb{R}$ un funcional subaditivo y
  positivamente homogeneo ($p(\lambda x)=\lambda p(x); \forall \lambda>0, x\in
  X)$. Sea $Y$ subespacio de $X$ de codimensión 1, y sea $f:Y\to \mathbb{R}$ 
  lineal tal que $f(x)\le p(x)$ $\forall x\in Y$. Entonces podemos extender $f$ 
  a $\tilde{f}(x):X\to \mathbb{R}$ tal que $\tilde{f}(x)\le p(x)$ $\forall x\in
  X$.
\end{lemma}

\begin{proof}
  $X=Y\oplus span \{x_0\} $. Solo hace falta definir $f(x_0)$ de forma que se
  mantenga la acotación. Cualquier extensión será de la forma
  \[
  \tilde{f}(x)=\tilde{f}(y+\alpha x_0)=f(y) + \alpha\tilde{f}(x_0)
  ,\] 

  y debe cumplirse:

  \begin{equation}
    \sup_{w\in Y} \{f(w)-p(w-x_0)\} \le \tilde{f}(x_0) \le 
    \inf_{z\in Y} \{-f(z)+p(z+x_0)\} 
  .\end{equation}

  Hace falta comprobar que ese intervalo no es vacío. Observamos que
  \[
  f(z)+f(w)=f(z+w)\le p(z+w) = p(z+x_0+w-x_0)\le p(z+x_0)+p(w-x_0), \forall
  z,w\in Y
  ,\]

  de donde $f(w)-p(w-x_0)\le -f(z)+(p(z+x_0)$ $\forall w,z\in Y$, de donde el
  intervalo de los posibles valores $\tilde{f}(x_0)$ es no vacío. Definimos
  entonces $\tilde{f}(x_0)$ en ese intervalo. queda solo ver que
  $\tilde{f}(y+\alpha x_0)\le p(y+\alpha x_0)$ $\forall y\in  Y, \alpha\in
  \mathbb{R}$. Basta trabajar por casos en el signo de $\alpha$: por ejemplo, si
  $\alpha\ge 0$:

  \[
  \tilde{f}(y+\alpha x_0)=
  f(y)+\alpha \tilde{f}(x_0)\le
  f(y) + \alpha\left( -f\left( \frac{y}{\alpha} \right)+p\left( \frac{y}{\alpha} +x_0\right) \right)=
  \alpha p\left( \frac{y}{\alpha} +x_0\right)=
  p(y+\alpha x_0)
  .\] 
\end{proof}

\begin{remark}
  Si $p$ es una seminorma (subaditiva y $p(\lambda x)=|\lambda|p(x)$) y
  $|f(x)|\le p(x)$ $\forall x\in Y$, entonces $|\tilde{f}(x)|\le p(x)$ $\forall
  x\in X$.
\end{remark}

\begin{remark}
  Si $p$ es una norma, esa condición se traduce en la acotación, y por lo tanto
  en la continuidad de la aplicación.
\end{remark}

\begin{theorem}[Hahn-Banach versión analítica]
  Sea $X$ un espacio de Banach, $Y$ subespacio cerrado de $X$, y
  $X=\overline{span} \{x_n:n=1,2,\ldots\} $. Considero $f:Y\to \mathbb{R}$ lineal
  y continua, entonces $|f(y)|\le \|f\|\|y\|$ y se puede construir una
  extensión con $|\tilde{f}(x)|\le \|x\|\|f\|$
\end{theorem}

\begin{proof}
  Si $Y_0:=Y,f_0=f$, defino $Y_n:=Y_{n-1}\oplus span \{x_n\}$ (si $x_n \in
  Y_{n-1}$ no hace falta extender, se tiene todo trivialmente), entonces para
  cada $n\in \mathbb{N}$ se puede extender $f_{n-1}$ a $f_n:Y_n\to \mathbb{R}$
  tal que $|f_n(x)|\le \|f_{n-1}\|\|x\|=\ldots=\|f\|\|x\|$.

  Defino entonces

  \begin{align*}
    \tilde{f}: Z:=\cup_{n=1}^\infty Y_n &\longrightarrow \mathbb{R} \\
    z &\longmapsto \tilde{f}(z) = f_n(z) \text{ si } z\in Y_n
  ,\end{align*}
  está bien definida porque los $Y_n$ son crecientes, y al extender no
  modificamos el valor en los $Y_n$ anteriores. Se tiene además que
  $|\tilde{f}(z)|\le \|f\|\|x\|$ $\forall z\in Z$. Pero además es lineal, así
  que es uniformemente continua y se puede extender a la clausura de $Z$ que es
  $X$.
\end{proof}

Trabajamos ahora en un espacio vectorial $E$ general.

\begin{definition}
  Cuando $\forall x$ existe $\lambda_0>0$ tal que $x\in \lambda A$ si
  $\lambda>\lambda_0$, se dice que $A$ es absorbente.
\end{definition}

\begin{definition}
  Dado $A$ absorbente con $0$ en el interior de $A$, defino $P_A(x):= \inf \{\lambda>0:x\in
  \lambda A\} $. A esta aplicación se le llama el funcional de Minkowski.
\end{definition}

\begin{proposition}
  Las siguientes propiedades son ciertas:
  \begin{enumerate}
    \item $P_A\ge 0$ y positivamente homogéneo.
    \item Si $A$ convexo, $P_A$ es subaditiva.
    \item $\{x:P_A(x)<1\} \subset A \subset \{x:P_A(x)\le 1\}  $.
  \end{enumerate}
\end{proposition}

\begin{proof}
  (Proposición 3.5.9., \cite{cascales2012}).
\end{proof}


\begin{theorem}[Mazur]
  Sea $E[\tau]$ un espacio vectorial topológico, $M\subset E$ una variedad
  afín y $A\subset E$ no vacío, abierto y convexo. Si $A\cap M= \emptyset$,
  entonces existe un hiperplano afín cerrado H en $E[\tau]$ tal que $A\cap
  H=\emptyset$ y $M\subset H$.
\end{theorem}

\begin{proof}
  $M=x_0+F$ con $F\subset E$ subespacio vectorial. Podemos suponer sin pérdida
  de generalidad que $0\in A$, pues se puede trasladar el problema. Como $A$ es
  abierto, $A$ es absorbente, y se comprueba además que $A=\{x\in E:P_A(x)<1\}
  $, y aplicando que $A\cap M=\emptyset$ $P_A(x_0+y)\ge  1, \forall y\in F$.
  Defino 
  \begin{align*}
    u: F\oplus span \{x_0\}  &\longrightarrow \to \mathbb{R} \\
    (y,\lambda x_0) &\longmapsto u((y,\lambda x_0)) = \lambda
  .\end{align*}
  
  $u$ es lineal, y $u(y+\lambda x_0) \le P_A(y+\lambda x_0)$ $\forall \lambda\in
  \mathbb{R},y\in F$, pues:

  \begin{itemize}
    \item si $\lambda < 0$, $u(y+\lambda x_0)=\lambda\le 0\le P_A(x_0+y)$;
    \item si $\lambda > 0$, $u(y+\lambda x_0)=\lambda \cdot 1 \le \lambda
      P_A(\frac{1}{\lambda}y+x_0)=P_A(y+\lambda x_0)$.
  \end{itemize}

  Aplicando a $u$ la forma analítica de Hahn-Banach, $u$ se extiende a
  $\tilde{u}:E\to \mathbb{R}$ tal que $\tilde{u}(x)\le P_A(x)$ para cada $x\in
  E$. Al ser $A$ un abierto, esa última desigualdad implica que $\tilde{u}$ es
  continua (comprobar el límite en $0$), y definiendo

  \[
  H=\{x\in E: \tilde{u}(x)=1\}
  ,\] 

  se tiene $M\subset H$ y $A\cap H=\emptyset$ ($x\in H\implies P_A(x)\ge
  \tilde{u}(x)=1\implies x\not\in A$).
\end{proof}

\begin{corollary}[1er Teorema de Separación]
  Sea $E[\tau]$ espacio vectorial topológico, $A$ y $B$ subconjuntos convexos no
  vacíos y abiertos, con $A\cap B= \emptyset$. Entonces existe un hiperplano $H$ 
  real cerrado que separa estrictamente $A$ y $B$, esto es, existe $f:E\to
  \mathbb{R}$ lineal y continua, $H=\{x\in E:f(x)=\xi\} $ y $A\subset \{x\in
  E:f(x)<\xi\}, B\subset \{x\in E:f(x)>\xi\} $.
\end{corollary}

\begin{proof}
  $A-B$ (diferencia algebraica, no de conjuntos) es convexo, abierto, no vacío y
  $0\not\in A-B$, por Mazur $\exists f:E\to \mathbb{R}$ lineal y continua con
  $0\in H=\{x\in E:f(x)=0\} $ y $H\cap (A-B)$, de donde $f(A-B)$ será conexo en
  $\mathbb{R}$, es decir, un intervalo, con $0\not\in f(A-B)$, luego $f$ separa
  estrictamente $A$ y $B$.
\end{proof}

\begin{corollary}[2o Teorema de Separación]
  Sea $E[\tau]$ e.v.t. localmente convexo, $K,F$ subconjuntos convexos disjuntos
  de $E$, con $K$ compacto y $F$ cerrado. Entonces existe un hiperplano real
  cerrado que separa estrictamente $K$ y $F$.
\end{corollary}

\begin{proof}
  Si $K$ es compacto, $\exists W$ entorno del origen convexo (por ser localmente
  convexo) tal que $(K+W)\cap (F+W)=\emptyset$ (ejercicio), bastando entonces
  aplicar el resultado anterior.
\end{proof}


\begin{theorem}[Hahn-Banach, versión geométrica]
  \label{th:h-b-geom}
  Dados dos conjuntos $A$ y $B$ convexos y cerrados, existe $H=\{x\in
  E:f(x)=\lambda\} $ que separa $A$ y $B$, donde $f:E\to \mathbb{R}$ es lineal y
  continua.
\end{theorem}
