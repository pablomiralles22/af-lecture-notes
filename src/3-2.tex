\subsection{Teorema de Baire.}

\begin{definition}
  Un conjunto es nunca-denso si el interior de su clausura es vacío.
\end{definition}

\begin{definition}
  Los conjuntos de primera categoría son las uniones numerables de conjuntos
  nunca-densos, y los de segunda categoría los que no son de primera.
\end{definition}

\begin{definition}
  Un espacio topológico se llama de Baire si la intersección de cualquier
  sucesión de abiertos densos es un conjunto denso.
\end{definition}

\begin{theorem}[Baire]
  \label{th:baire}
  Si $(M,d)$ es un espacio métrico completo, entonces $M$ es de Baire.
\end{theorem}
\begin{proof}
  (p. 263, \cite{cascales2012}).

  Sea $(G_n)_n$ una sucesión de abiertos densos. Para demostrar que $\cap_{n\in
  \mathbb{N}} G_n$ es denso, veamos que para cualquier abierto $\emptyset\neq
  V\subset M$ se tiene que $V\cap \left( \cap _{n\in \mathbb{N}}G_n \right) \neq
  \emptyset$. Construyo inductivamente sucesiones $(x_n)_n\subset M$,
  $(r_n)_n\subset \mathbb{R}^+$ de forma que $B[x_n,r_n] \subset G_n\cap
  B(x_{n-1},r_{n-1}) \subset G_n \cap G_{n-1}\cap \ldots\cap G_1 \cap V$ y $r_n
  < \frac{1}{n}$.

  \begin{itemize}
    \item Como $G_1$ denso en $M$, $\exists x_1 \in G_1\cap V\neq \emptyset$, y
      por ser abierto $\exists r_1 \in (0,1)$ con $B[x_1,r_1] \subset V\cap
      G_1$.
    \item Como $G_n$ denso en $M$, $\exists x_n \in G_n\cap
      B(x_{n-1},r_{n-1})\neq \emptyset$, y por ser abierto $\exists r_n \in
      (0,\frac{1}{n})$ con $B[x_n,r_n] \subset G_n\cap B(x_{n-1},r_{n-1})
      \subset G_n \cap G_{n-1}\cap \ldots\cap G_1 \cap V$.
  \end{itemize}

  $(x_n)_n$ es de Cauchy, pues para cada $m>n$  $x_m \in B(x_m,r_m)\subset
  B(x_{m-1},r_{m-1})\subset \ldots\subset B(x_n,r_n)$, y $r_n\to 0$. Como el
  espacio es completo existe $x=\lim_{n \to \infty} x_n$, debe ser $x\in
  B[x_n,r_n]$ para cada $n\in \mathbb{N}$, pues ya hemos visto que $\forall m>n$ 
  $x_m\in B(x_n,r_n)\subset B[x_n,r_n]$, y este es cerrado, basta tomar límites
  en $m$. Se tiene por construcción entonces que $x\in G_n \cap G_{n-1}\cap
  \ldots\cap G_1 \cap V$ para cada $n\in \mathbb{N}$, y por lo tanto $x\in V\cap
  \left( \bigcap_{n\in \mathbb{N}} G_n \right) $
\end{proof}

\begin{corollary}
  Si $X$ es un espacio de Banach, entonces su dimensión algebraica o es finita o
  es no numerable.
\end{corollary}

\begin{proof}
  (p. 264, \cite{cascales2012}).
\end{proof}

