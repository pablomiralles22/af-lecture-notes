\subsection{Problemas variacionales cuadráticos.}

\begin{theorem}[Teorema principal de los problemas variacionales cuadráticos]
  \label{th:prob-var-cuad}
  Sea $H$ espacio de Hilbert real y $B$ una forma bilineal simétrica, acotada y
  fuertemente positiva definida en $H$. Sea $b$ una forma lineal continua en $H$ 
  y sea $F:H\to \mathbb{R}$ definida mediante

  \[
  F(x):=\frac{1}{2}B(x,x)-b(x)
  .\] 
  Entonces:
  \begin{enumerate}
    \item Es condición necesaria y suficiente para que $F$ alcance su mínimo en
      $w \in H$ que se verifique
      \[
        B(w,y)=b(y)
      \] 
      para cada $y\in H$.

    \item La función real $F(x)$ alcanza un mínimo absoluto en $H$, que además
      es único.
  \end{enumerate}
\end{theorem}
