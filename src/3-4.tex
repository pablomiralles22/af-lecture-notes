\subsection{Teorema de la aplicación abierta y de la gráfica cerrada.}

\begin{definition}
  Sea $X$ un espacio normado y $A\subset X$ :
  \begin{enumerate}
    \item Se dice que $A$ es CS-compacto si para toda sucesión $(x_n)_n\subset
      A$ y cualquier sucesión $(\lambda_n)_n\subset [0,1]$ tal que
      $\sum_{n=1}^{\infty} \lambda_n=1$ se verifica que la serie
      $\sum_{n=1}^{\infty} \lambda_n x_n$ converge a un punto de $A$.
    \item Se dice que $A$ es CS-cerrado si para toda sucesión $(x_n)_n\subset
      A$ y cualquier sucesión $(\lambda_n)_n\subset [0,1]$ tal que
      $\sum_{n=1}^{\infty} \lambda_n=1$ y para la cual la serie
      $\sum_{n=1}^{\infty} \lambda_n x_n$ converge a un punto, se verifica que
      dicho punto está en $A$.
  \end{enumerate}
\end{definition}

\begin{proposition}
  \label{prop:open-map-1}
  Sea $X$ espacio normado y $A\subset X$:
  \begin{itemize}
    \item Si $X$ es de Banach su bola unidad es CS-compacta.
    \item Si $A$ es cerrado y convexo, entonces $A$ es CS-cerrado.
    \item Si $A$ es CS-compacto, entonces $A$ es CS-cerrado y acotado. Si $X$ es
      de Banach, entonces el recíproco también es cierto.
  \end{itemize}
  (Prop 3.4.2., \cite{cascales2012})
\end{proposition}

\begin{proposition}
  \label{prop:open-map-2}
  Sean $X,Y$ espacios normados y $T:X\to Y$ lineal y continua. Si $A\subset X$ 
  es CS-compacto entonces $T(A)$ es CS-compacto.
  (Prop 3.4.3., \cite{cascales2012})
\end{proposition}

\begin{proposition}
  \label{prop:open-map-3}
  Sea $X$ espacio normado y $A\subset X$ CS-cerrado. Entonces $A$ y
  $\overline{A}$ tienen el mismo interior.
  (Prop 3.4.4., \cite{cascales2012})
\end{proposition}


\begin{theorem}[De la aplicación abierta]
  \label{th:open-map}
  Sean $X$ un espacio de Banach e $Y$ un espacio normado. Sea $T:X\to Y$ lineal
  y continua tal que $T(X)$ es de segunda categoría en $Y$. Entonces $T$ es una
  aplicación sobreyectiva y abierta, siendo además $Y$ un espacio de Banach.
\end{theorem}

\begin{proof}(Teo. 3.4.5., \cite{cascales2012})

  Para ver que $T$ es abierta y sobreyectiva basta comprobar que $T(B_X)\supset
  rB_Y$ para cierto $r>0$.

  Como $B_X$ es CS-compacto (\autoref{prop:open-map-1}.(i)), $T(B_X)$ también lo es
  (\autoref{prop:open-map-2}), así que $T(B_X)$ es CS-cerrado
  (\autoref{prop:open-map-1}.(iii)).

  Como $T(X)=\bigcup_{n} nT(B_x)$ es de segunda categoría en $Y$, y las
  homotecias son homomorfismos,  $\overline{T(B_X)}$ tiene interior no vacío,
  que coincide con el interior de $T(B_X)$ (\autoref{prop:open-map-3}). Puedo
  tomar entonces $y_0\in Y,r>0$ tales que $B_Y(y_0,r)\subset T(B_X)$. Por la
  simetría de las bolas, $B_Y(-y_0,r)\subset T(B_X)$, y por lo tanto:

  \begin{equation}
    \label{eq:open-map-ball-equality}
    B_Y(0,r) \subset \frac{1}{2}B_Y(-y_{0},r)+\frac{1}{2}B_Y(y_0,r)
    \subset \frac{1}{2}T(B_X)+\frac{1}{2}T(B_X)\subset T(B_X)
  \end{equation}
  como queríamos ver. Falta solo comprobar que $Y$ es completo entonces, para lo
  cual es suficiente ver que si $(y_n)_n\subset Y$ y $\sum_n \|y_n\|<\infty$
  entonces $\sum_ny_n$ converge.

  Por (\autoref{eq:open-map-ball-equality}), para cada $n$ podemos tomar $x_n
  \in B_X$ con $Tx_n=\frac{r}{2}\frac{y_n}{\|y_n\|}$, así que haciendo
  $z_n=\frac{2}{r}\|y_n\|x_n$ se da $Tz_n=y_n$ y $\|z_n\|\le
  \frac{2}{r}\|y_n\|$. Pero entonces $\sum_n \|z_n\|<\infty$, y existe
  $x:=\sum_n x_n$ por la completitud de $X$. Finalmente, por la continuidad de
  $T$, $\sum_n y_n$ es convergente a $Tx$.



\end{proof}

Una versión más particular, y seguramente la que pregunte en el oral:

\begin{corollary}
  \label{cor:open-map}
  Sean $X,Y$ espacios de Banach y $T:X\to Y$ lineal y continua, entonces $T$ es
  sobreyectiva si y solo si es abierta.
\end{corollary}

\begin{definition}
  Una aplicación $f:M_1\to M_2$, donde $M_1,M_2$ son espacios topológicos
  Hausdorff, se dice que tiene gráfica cerrada si su gráfica, es decir, el
  conjunto
  \[
  Graf(f)=\{(t,f(t)):t\in M_1\} 
  \] 
  es cerrado en el espacio producto $M_1\times M_2$.
\end{definition}

\begin{theorem}[De la gráfica cerrada]
  Sean $X,Y$ espacios de Banach y $T:X\to Y$ lineal. Entonces $T$ es continua si
  y solo si $TA$ tiene gráfica cerrada.
\end{theorem}

\begin{proof}

  Algunas observaciones:

  \begin{enumerate}
    \item Como $T$ es lineal, $Graf(T)$ es un espacio vectorial.
    \item Considero $P_1,P_2$ las proyecciones canónicas en $X \times Y$, que
      son lineales y continuas en la topología producto (dada por $\|\cdot \|$).
    \item Llamo $\hat{P}_i=P_i|_{Graf(T)}$, es claro que $\hat{P}_1$ es
      biyectiva.
  \end{enumerate}

  \begin{description}
    \item[$\mathbf{\implies}$] Sea $((x_n,Tx_n))_n$ sucesión en $Graf(T)$ que
      converge a $(x,y)\in X \times Y$. Como las proyecciones son continuas,
      $x_n=P_1(x_n,Tx_n)\to P_1(x,y)=y$, y $Tx_n=P_2(x_n,Tx_n)\to P_2(x,y)=y$.

      Como $T$ es continua, $Tx = \lim_{n \to \infty} Tx_n=y$, así que
      $(x,y)=(x,Tx)\in Graf(T)$, con lo que $Graf(T)$ cerrado.

    \item[$\mathbf{\impliedby}$] Por la observación 1 y bajo la hipótesis de que
      $Graf(T)$ es cerrado en un espacio de Banach, $Graf(T)$ es también un
      espacio de Banach.

      Considero la aplicación $S:X\to Graf(T)$, $x\mapsto (x,Tx)$. $S$ es
      claramente biyectiva, con inversa $\hat{P}_1$. Como $P_1$ es lineal y
      continua $\hat{P}_1$ también, así que por (\autoref{cor:open-map}) es
      abierta y $S$ es continua. Se tiene entonces que $T=P_2\circ S$ también lo
      es, pues es composición de continuas.
  \end{description}

\end{proof}

