\subsection{Operadores compactos.}

\begin{definition}
  Dado $X$ espacio topológico, $Y\subset X$ es relativamente compacto si su
  clausura es compacta.
\end{definition}

\begin{definition}
  Un operador es compacto si la imagen de la bola unidad es relativamente
  compacta.
\end{definition}

\begin{remark}
  En espacios métricos, y en concreto en espacios normados, la compacidad
  relativa equivale a que cualquier sucesión contenga una subsucesión
  convergente (en el espacio $X$, no en $Y$).
\end{remark}

\begin{lemma}
  \label{le:finite-range-adjoint}
  Si $K:H\to H$ es un operador de rango finito en $H$ Hilbert, entonces su
  adjunto es de rango finito (p. 162, \cite{cascales2012}).
\end{lemma}
\begin{proof}
  Si el rango de $K$ es $span\{v_1,\ldots,v_n\}$, con los $v_i$ ortonormales, y
  $P_i$ es la proyección de $H$ en $span \{v_i\}$, entonces $P_i\circ K$ son
  formas lineales continuas, $\exists ! u_i$ tal que $P_i\circ K x =
  \dotp{x}{u_i}v_i$ para cada $x\in H$. Se tiene entonces:

   \[
  Kx = \sum_{i=1}^{n} \dotp{x}{u_i} v_i
  \] 
  para cada $x\in H$. Pero entonces para $x,y\in H$ arbitrarios:

  \[
  \dotp{x}{K^*y}=\dotp{Kx}{y}=\sum_{i=1}^{n} \dotp{x}{u_i}\dotp{v_i}{y}=
  \dotp{x}{\sum_{i=1}^{n} \dotp{y}{v_i}u_i}
  ,\] 
  y $K^*y=\sum_{i=1}^{n} \dotp{y}{v_i}u_i$ para cada $y\in H$, tiene rango
  finito.
\end{proof}

\begin{theorem}
  \label{th:operator-compactness-properties}
  Sea $H$ Hilbert separable, $T:H\to H$ lineal y continua, entonces:

  \begin{enumerate}
    \item $S:H\to H$ compacto implica que $S\circ T,T\circ S$ compactos.
    \item $T_n:H\to H, n=1,2,\ldots$ compactos y $\lim_{n \to \infty}
      \|T_n-T\|=0$ implica $T$ compacto.
    \item $T$ compacto implica que $\exists T_n:H\to H$ operadores de rango
      finito con $\lim_{n \to \infty} \|T_n-T\|=0$.
    \item $T$ compacto si y solo si  $T^*$ compacto.
  \end{enumerate}
\end{theorem}
\begin{proof}\hfill
  \begin{enumerate}
    \item Sea $(x_k)_k\subset B_H$, $(Sx_k)_k$ tiene una subsucesión convergente
      $Sx_{k_i}\to y$, por continuidad $TSx_{k_i}\to Ty$. Considero ahora
      $(Tx_k)_k$, como $T$ operador $\|T\|<\infty$, $\left( \frac{Tx_k}{\|T\|}
      \right)_k\subset B_H $, luego $\left( \frac{STx_k}{\|T\|} \right)_k$ tiene
      una subsucesión convergente $\frac{STx_{k_i}}{\|T\|}\to y$, con lo que
      $STx_{k_i}\to \|T\|y$.

    \item Sea $(f^0_k)_k\subset B_H$, para cada $n\in \mathbb{N}$, como $T_n$ 
      compacto, $\exists (f^{n}_k)_k$ subsucesión de $(f^{n-1}_k)_k$ tal que
      $(T_n f^n_k)_k$ es convergente. Considero ahora $(Tf_k^k)_k$, veamos que
      es de Cauchy (y por lo tanto convergente en H completo).

      \begin{align*}
        \|Tf_k^k - Tf_l^l\| & \le  \|Tf_k^k - T_mf_k^k\| + \|T_mf_k^k - T_mf_l^l\| + \|T_mf_l^l - Tf_l^l\| \\
                            & \le  \|T - T_m\|(\|f_k^k\|+\|f_l^l\|) + \|T_mf_k^k - T_mf_l^l\| \\
                            & \le  \|T - T_m\|\cdot 2 + \|T_mf_k^k - T_mf_l^l\| \\
      .\end{align*}

      Dado $\varepsilon>0$, tomo $m_0$ tal que
      $\|T-T_{m_0}\|<\frac{\varepsilon}{4}$, y como para $k\ge m_0$ $(f^k_k)_k$ es
      subsucesión de $(f_k^{m_0})_k$, cuya imagen por $T_{m_0}$ converge, puedo
      tomar $l,k$ suficientemente grandes para que $\|T_{m_0}f_k^k -
      T_{m_0}f_l^l\|<\frac{\varepsilon}{2}$.

    \item Tomo $\{e_k\}_1^\infty$ base hilbertiana. Para $H_n:=span
      \{e_1,e_2,\ldots,e_n\}$, $P_n:=P_{H_n}$, $P_n+Q_n=Id$, $P_n\circ T$ tiene
      rango en $H_n$ finito-dimensional. Basta ver que $\|T-P_n\circ
      T\|=\|Q_n\circ T\|\to 0$.

      $\|Q_nT\|=\sup \{\|Q_nT f\|: \|f\|\le 1\} =\sup \{\|Q_ng\|:g\in T(B_H)\}
      $. Pero $T(B_H)$ es relativamente compacto, y $\{Q_n\}_{n\in \mathbb{N}}$ 
      es equicontinua, $\|Q_n\|=1$ para cada $n\in \mathbb{N}$, así que por el
      teorema de Ascoli la convergencia puntual y la uniforme son equivalente.
      Basta ver entonces que $\|Q_ng\|\to 0$ para cada $g\in T(B_H)$.
       \[
       \|Q_ng\|^2=\sum_{i=n+1}^{\infty} |\dotp{g}{e_i}|^2\to 0
       ,\] 
       pues la serie $\sum_{i=1}^{\infty} |\dotp{g}{e_i}|^2=\|g\|$ converge.

     \item Para la necesidad basta aplicar (\ref{le:finite-range-adjoint}) y el
       apartado (2), junto con que $(T_n-T)^*=T_n^*-T^*$. La suficiencia se
       sigue de que  $T=T^{**}$.
  \end{enumerate}
\end{proof}

\begin{remark}
  Si $T:H\to H$ es un operador de rango finito, entonces es compacto, pues la
  imagen está contenida en una bola de un espacio de dimensión finita, que es
  compacta, 
\end{remark}

\begin{proposition}
  \label{prop:diagonal-operator-properties}
  Sea $H$ Hilbert separable, $\{e_k\}_1^\infty$ base ortonormal,
  $(\lambda_k)_k \subset \mathbb{C}$ acotada. Defino $Te_k:=\lambda_k e_k$,
  luego en general $Tx=\sum_{n=1}^{\infty} \lambda_k \dotp{x}{e_k}e_k$.
  Entonces:

  \begin{enumerate}
    \item $\|T\|=\sup \{|\lambda_k|:k=1,2,\ldots\} $.
    \item $T^* e_k=\overline{\lambda_k}e_k$.
    \item $T=T^* \iff (\lambda_k)_k\subset \mathbb{R}$.
    \item T es un proyector si y solo si $(\lambda_k)_k\subset \{0,1\} $.
    \item T es compacto si y solo si  $\lim_{k \to \infty} \lambda_k=0$.
  \end{enumerate}
\end{proposition}

\begin{proof}\hfill
  \begin{enumerate}
    \item Para $\|x\|\le 1$, 
      \begin{align*}
        \|Tx\| & \le \sum_{n=1}^{\infty} |\lambda_k| |\dotp{x}{e_k}| \|e_k\| \\
               & \le \sup \{|\lambda_k|:k=1,2,\ldots\} \sum_{n=1}^{\infty} |\dotp{x}{e_k}| \\
               & = \sup \{|\lambda_k|:k=1,2,\ldots\} \|x\| \\
               & = \sup \{|\lambda_k|:k=1,2,\ldots\}
      .\end{align*}
    \item $\dotp{T^*e_k}{e_i}= \overline{\dotp{e_i}{T^*e_k}}=
      \overline{\dotp{Te_i}{e_k}}= \overline{\dotp{\lambda_i
      e_i}{e_k}}=\overline{\lambda_i} \delta_{i,k}$.

    \item Evidente por el apartado anterior.
    \item Si la definición de proyector es que $\exists M\subset H$ subespacio
      cerrado tal que $T=P_M$ es evidente.
    \item Comienzo por la necesidad. Si $\lim_{k \to \infty} \lambda_k \neq 0$,
      puedo tomar una subsucesión $(\lambda_{k_i})_i$ con $|\lambda_{k_i}|\ge
      \varepsilon$ para cierto $\varepsilon>0$. Si $i\neq j$,
      $\|Te_{j_i}-Te_{k_j}\|=\sqrt{|\lambda_{k_i}|^2+|\lambda_{k_j}|^2}\ge \varepsilon$,
      luego no puede tener ninguna subsucesión de Cauchy ni convergente.

      Para la suficiencia, por el apartado (2) de
      (\ref{th:operator-compactness-properties}) y que los operadores de rango
      finito son compactos, basta ver que $\|P_nT-T\|=\|Q_nT\|$, donde
      $H_n:=span \{e_1,e_2,\ldots,e_n\}$, $P_n:=P_{H_n}$, $P_n+Q_n=Id$. Para
      $x\in H$ con $\|x\|\le 1$:

      \begin{align*}
        \|Q_nTx\|^2 & =\sum_{i=n+1}^{\infty} |\lambda_k| |\dotp{x}{e_k}|\\
                    & \le \sup \{|\lambda_k|: k>n\}\sum_{i=n+1}^{\infty} |\dotp{x}{e_k}|\\
                    & \le \sup \{|\lambda_k|: k>n\}\|x\|\\
                    & \le \sup \{|\lambda_k|: k>n\} \to 0
      ,\end{align*}

      cuando $n\to \infty$, uniformemente en $B_H$.
  \end{enumerate}
\end{proof}


\begin{example}
  $k:[a,b]\times[a,b]\to \mathbb{R}$ continua,
  \begin{align*}
    K: L^1[a,b] &\longrightarrow C[a,b] \\
    f &\longmapsto K(f)(s) = \int_{a}^{b} k(s,t)f(t)dt 
  ,\end{align*}
  es compacto (p.178, \cite{cascales2012}).
\end{example}

\begin{example}
  Si $\Omega\subset \mathbb{R}^n$ suficientemente regular, $\forall f\in
  L^2(\Omega)$ existe un único $u\in H=\overline{C_0^\infty(\Omega)}$ con
  $\dotp{u}{v}=\int_\Omega uv + \int_\Omega \sum u_jv_j$.

  \begin{align*}
    S: L^2(\Omega) &\longrightarrow H \xhookrightarrow{} L^2(\Omega) \\
     f&\longmapsto u 
  \end{align*}

  es compacto por el teorema de Rellich (\ref{th:rellich}).
\end{example}

\begin{theorem}[Rellich]
  \label{th:rellich}
  Sea $\Omega\subset \mathbb{R}^n$ abierto con frontera suficientemente regular,
  y $\mathcal{F}\subset L^2(\Omega)$ tal que para cierto $K>0$, $\forall g\in
  L^2(\Omega)$:
  \begin{enumerate}
    \item $g_j=\frac{\partial g}{\partial x_j} \in L^2(\Omega)$ $\forall
      j=1,\ldots,n$.
    \item $\|g\|_{L^2},\|g_j\|_{L^2}\le K$ $\forall j=1,\ldots,n$.
  \end{enumerate}

  Entonces $\mathcal{F}$ es relativamente compacto en $L^2(\Omega)$.
\end{theorem}
\begin{proof}
  Tarea.
\end{proof}

