\subsection{Inversión de operadores. Espectro.}
\begin{definition}
  Un operador es una aplicación lineal y continua.
\end{definition}

\begin{theorem}[Von-Neumann]
  \label{th:Von-Neumann}
  Sea $X$ un espacio de Banach, $K\in \mathcal{L}(K)$ invertible y $L:=K-A$. Si
  $\|A\|<\|K^{-1}\|^{-1}$ entonces $L$ es invertible.
\end{theorem}

\begin{proof}
  Estudio primero el caso en el que $K=Id$, se trata de ver que  $Id-B$ con
  $\|B\|<1$ es invertible. Defino $S=\sum_{n=0}^{\infty} B^n$, refiriéndose el
  exponente a la composición. $\sum_{n=0}^{\infty} \|B^n\|\le
  \sum_{n=0}^{\infty} \|B\|^n$ convergente, así que la serie converge y $S\in
  \mathcal{L}(X)$, que es de Banach por serlo $X$.

  Veamos que $S=(Id-B)^{-1}$.

  \begin{itemize}
    \item $B\circ S=B\circ \sum_{n=0}^{\infty} B^n
      \underbrace{=}_{\circ \text{ continua}}=\sum_{n=0}^{\infty}
      B^{n+1}=S-Id\implies(Id-B)\circ S=Id$.
    \item Análogamente $S\circ (Id-B)=Id$.
  \end{itemize}

  En el caso general, $K-A=K\circ (Id-K^{-1}\circ A)$, $\|K^{-1}\circ A\|\le
  \|K^{-1}\|\|A\|<1$ por hipótesis, así que aplicando el primer caso
  $Id-K^{-1}A$ es invertible. Como $K$ es invertible y la composición de
  invertibles es invertible ya lo tenemos.
\end{proof}

\begin{remark}
  \label{rem:inverse-formula}
  $(K-A)^{-1}=(K\circ (Id-K^{-1}A))^{-1}=(Id-K^{-1}\circ A)^{-1}\circ
  K^{-1}=\left( \sum_{n=0}^{\infty} (K^{-1}\circ A)^n \right) \circ
  K^{-1}=\sum_{n=0}^{\infty} (K^{-1}\circ A)^n K^{-1}$.
\end{remark}

\begin{definition}
  Sea $M:X\to X$ un operador, se definen:
  \begin{itemize}
    \item $\rho(M)=\{\lambda\in \mathbb{C}:(\lambda Id-M) \text{ es
      invertible}\} $.
    \item $\sigma(M)=\mathbb{C}\setminus \rho(M)$ es el espectro del operador.
  \end{itemize}
\end{definition}

\begin{theorem}
  Dado $M$ un operador:
  \begin{enumerate}
    \item $\rho(M)$ es abierto.
    \item $\phi:\rho(M)\to  \mathcal{L}(X)$, dada por $\phi(\lambda)=(\lambda
      Id-M)^{-1}$ es analítica.
  \end{enumerate}
\end{theorem}

\begin{proof}
  \begin{enumerate}
    \item $\lambda\in \rho(M)\implies K=\lambda Id-M$ es invertible. Para $h\in
      \mathbb{C}$ con $|h|<\|K^{-1}\|^{-1}$, $(\lambda-h)Id-M=K-hId$, con
      $\|hId\|=|h|\|Id\|=|h|<\|k^{-1}\|^{-1}$, así que es invertible y
      $\lambda-h\in \rho(M)$.
    \item Se tiene además por (\ref{rem:inverse-formula}) lo siguiente:

      \[
        \phi(\lambda-h)=((\lambda-h)Id-M)^{-1}=\sum_{n=0}^{\infty} ((\lambda Id-M)^{-1}\cdot
        h)^n(\lambda Id-M)^{-1}=\sum_{n=0}^{\infty} ((\lambda
        Id-M)^{-1})^{n-1}h^n
      ,\] 
      luego es analítica.
  \end{enumerate}
\end{proof}

\begin{theorem}[Gelfand]
  $\forall M\in \mathcal{L}(X)$, $\sigma(M)$ es un compacto no vacío.
\end{theorem}

\begin{proof}
  $\rho(M)$ abierto implica $\sigma(M)$ cerrado, veamos que es acotado. Si
  $|\xi|> \|M\|$, veamos que $\xi \not\in \sigma(M)$, bastando aplicar
  (\ref{th:Von-Neumann}) para $K=\xi  Id, A=M$.

  Además, aplicando (\ref{rem:inverse-formula}):

  \[
    \phi(\xi )=(\xi Id-M)^{-1}=\sum_{n=0}^{\infty} M^n\cdot \xi^{-(n+1)}
  .\] 

  Si fuese vacío $\sigma(M)$, entonces $\phi$ sería entera, tendría dominio
  $\mathbb{C}$. Se tiene también

  \[
  \|\phi(\xi)\|\le \sum_{n=0}^{\infty} \|M\|^n |\xi|^{-(n+1)}=
  \frac{1}{|\xi|-\|M\|}\to 0
  \] 
  cuando $|\xi|\to \infty$. Por el teorema de Liouville $\phi$ es
  constante, pero $\phi'(\xi)=-(\xi Id - M)^{-2} \neq 0$, contradicción.
\end{proof}

\begin{proposition}
  Si $M$ es un subespacio cerrado de  $H$, $\dotp{P_Mx}{y}=\dotp{x}{P_My}$ para
  cada $x,y\in H$.
\end{proposition}

\begin{proof}
  Ejercicio.
\end{proof}
